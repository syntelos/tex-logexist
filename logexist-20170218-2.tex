\input preamble

The inner consciousness is remote to the outer consciousness of the
observer.  The inner consciousness is to this logical existentialism
what the quantum foam is to space time.  It is the fuzzy boundary at
the limit of resolution.  Carl Jung demonstrated the derivation of
speech from the intimacy of the inner consciousness in his
autobiography.  Sigmund Freud mapped the domain in a model that served
his disciplines of observation and interaction.  The purpose here is
to develop elements of thought that may identify and characterize the
inner consciousness for the benefit of its knowledge and experience.

That the inner consciousness emerges within the womb has a cause of
knowledge with the dominance of the subjective nature of the inner
consciousness over the objective nature of the outer consciousness
known to childhood, and the existence of the trauma of the passage of
canal birth within the conscious human experience.  The inner
consciousness is first described as subjective, emotional, and
spiritual.

The inner consciousness is emotion and character and the invariance
within one's moral fabric.  It is a genetic gift.  It is asexual
sexuality, force of wisdom, and power of love.  It is the commitment
of partner and parent.  It is spirit and spirituality.  It is the
semantic core within the human existence, the depth of meaning beyond
pleasure and pain.  It is joy and prayer.  It is the divinity of the
self.  And, ideally, the center of one's intimate silences.  

The inner consciousness, $\sigma$, perceives the state of the outer
($\kappa$) and social ($\gamma$) consciousnesses into its state of
spiritual being.  $$ \sigma \leftarrow \phi\kappa + \phi\gamma
+ \sigma. $$ It is the contemporaneous spirituality which moves its
marginal variance with the perception of outer and social
consciousness.  $$\displaylines{ \kappa \leftarrow \phi\sigma
+ \phi\gamma + \kappa, \cr \gamma \leftarrow \phi\kappa + \phi\sigma
+ \gamma, \cr \sigma \leftarrow \phi\kappa + \phi\gamma + \sigma.}$$
This temporal system of linear equations describes the dynamical
system of consciousness in the perceptions across the interior
structures that maintain the coherence of the self while the system
may become unhealthy.

\bye
