\input preamble

The spiritual state of being of the individual, $\psi$, has a
corollary in the consciousness of the individual, $\kappa$.

On a reduced (lower) level of abstraction the outer consciousness,
$\kappa$, is contrasted to (differentiated from) the inner
consciousness, $\sigma$, and the social consciousness, $\gamma$.

In an epistemology of the human existence with respect to mind and
spirit, a metaphysical logic, the definition of abstraction is faced
with the challenge inherent in the distinction of ``metaphysical
logical object'' from ``metaphysical term of logic''.  The former is
an element of thought and the latter is an element of logic.  The
metaphysical logic is a tool employed to communicate an exposition of
thought concerning the knowledge of human existence.  And the intent
of ``logical object'' is not exclusive of the broader interpretation
as ``object of'' (communication or thought).

A center of consciousness is capable of directing or navigating the
human existence in its own terms.  The outer consciousness, $\kappa$,
is that which we would intend by the use of the term ``sense'' or
``senses'', as well as that which we would typically intend by the use
of the term ``intellect'' or ``mind''.  The inner consciousness,
$\sigma$, is the subjective and emotional bed of human existence.  We
emerge within our mother's womb within the inner consciousness, and we
sleep within the inner consciousness.  The social consciousness,
$\gamma$, is a sense of others and space and culture that includes the
so-called ``god node''.

From this point we can realize the existential frame operator, $\phi$,
as analogous to intellectual abstraction in the
following. $$\displaylines{ \kappa \leftarrow \phi\kappa, \cr \kappa \leftarrow \phi\sigma, \cr \sigma
\leftarrow \phi\kappa, \cr \kappa \leftarrow \phi\gamma, \cr \gamma \leftarrow \phi\sigma, \cr \sigma \leftarrow \phi\gamma, \cr \phi \hookleftarrow \kappa + \sigma + \gamma.}$$ In a brief review, intellect is conscious of spirit, spirit is conscious of intellect, and then (in high abstraction) the abstraction of the
contextual frame is analogous to consciousness.



\bye
